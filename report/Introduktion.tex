% -*- coding: utf-8 -*-

% NOTER TIL DETTE AFSNIT:
% - Motivation
%   - Hvorfor CUDA? (meget udbredt, mulige masseproduktionsfordele).
%   - Hvorfor vector/scan? (nævn evt. at det kunne være vejen til
%     at impelmentere en backend for det der vector/scan sprog,
%     eller i hvert fald en undersøgelse om det kunne betale sig).
% - Mål
%   - Er vector/scan-modellen realistisk på CUDA?
%   - Hvor arbejdskrævende er det at benytte CUDA?
%   - Hvad skal man vide for at benytte CUDA?
%   - Hvordan optimerer man på CUDA?
% - Fokus
% - Læringsmål

\section{Introduktion}
I de sidste par år er man ved at have nået en hardwaremæssig grænse for hvor
hurtigt sekventiele programmer kan afvikles, og man er derfor begyndt at fokusere på at eksekvere programmerne
parallelt. Dette har allerede længe været tilfældet for GPU'er (Graphical Prosessing Units), som typisk kommer med mange kerner, og som samtidigt ligger i en overskuelig prisklasse. Det er derfor interessant 
at undersøge hvordan man bedst udnytter sådanne processorer til generelle algoritmer. 

Der findes algoritmer der er åbentlyst paralleliserbare (map), men også mindre åbentlyse algoritmer, som at 
summere værdierne i en vektor, kan gøres effektivt parallelt. Scan er generalisering af dette, der anvender en associativ operator på elementerne i vektoren, og efterlader præfix-summen op til hvert element på elementets position. En måde at implementere disse på er vha. vektor-scan modellen, som tillader skabelsen af mange primitiver der kan bruges til at udvikle generelle algoritmer, samtidigt med at de er designet til at kunne eksekveres i et paralelt miljø - som f.eks. NVIDIAs CUDA platform.

\subsection{Mål}

Vi vil i dette projekt undersøge om det kan betale sig at implementere vector-scan modellen på CUDA, med det formål at demonstrere en implementation af Scan på NVIDIAs CUDA platform. Vi vil i dette projekt desuden gøre det til vores mål at lære at programmere på CUDA platformen, samt at forstå scan-vector modellen, og videregive de erfaringer vi har gjort os - med særlig fokus på optimering.

For at give en kvantitativ indikator af hvor godt vores projekt har lykkedes, vil vi sammenligne vores egne implementationer med CUDA's implementation af scan, samt en sekventiel version af algoritmen.\\
