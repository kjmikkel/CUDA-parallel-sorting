% -*- coding: utf-8 -*-

% NOTER TIL DETTE AFSNIT:
% - Put referencer til kildekodefilerne (split, copy, radix, quicksort, kernel)
% - Hvad svarer de enkelte ting til i vector-scan-modellen/Blellochs bog?
% - Hvordan bliver API'et brugt? Er det (teoretisk) optimalt?
% * Map, scan, etc. bør nok fjernes herfra og beskrives i Paracuda impl.

\section{Implementation}
\label{implementation}

\subsection{Split}

Vi har implementeret split oven på vores bibliotek (altså uden at bruge
CUDA direkte). Vores implementation svarer til det følgende:

\[
\begin{array}{l}
\mbox{split} (f, v) = \\
\mathtt{let\ } n = \mbox{map} (\neg, f) \mathtt{\ in\ } \\
\mathtt{let\ } (l, sum) = \mbox{scan} (+, 0, n) \mathtt{\ in\ } \\
\mathtt{let\ } c = \mbox{copy} (sum, \#v) \mathtt{\ in\ } \\
\mathtt{let\ } (t, \_) = \mbox{scan} (+, 0, f) \mathtt{\ in\ } \\
\mathtt{let\ } r = \mbox{map} (+, \mbox{zip}(c, t)) \mathtt{\ in\ } \\
\mathtt{let\ } g = \lambda (a, b, c). \mathtt{\ if\ } a \mathtt{\ then\ } c \mathtt{\ else\ } b \mathtt{\ in\ } \\
\mathtt{let\ } p = \mbox{map} (g, \mbox{zip} (f, l, r)) \mathtt{\ in\ } \\
\mbox{permute} (p, v) \\
\end{array}
\]

Hvor $\mbox{zip}$ foretages ved at pege pointerene i en tuppel
hen på de relevante arrays. $\#v$ er antallet af elementer i $v$.
Derudover foretages der allokering og deallokering
af hukommelse. Koden for delalgoritmerne, datastrukturerne og operatorerne er:

\begin{verbatim}
PARACUDA_SINGLE(int)

PARACUDA_OPERATOR(plus, void, (int* r, int* a, int* b), ({
    *r = *a + *b;
}))

PARACUDA_OPERATOR(zero, void, (int* r), ({
    *r = 0;
}))

PARACUDA_SCAN(plus_scan, plus, zero, int)

PARACUDA_OPERATOR(negate, void, (int* r, int* a), ({
    *r = !*a;
}))

PARACUDA_MAP(negate_map, negate, int, int)

PARACUDA_STRUCT_3(split_t, split_vector_t,
    int, flags, 
    int, left,
    int, right
)

PARACUDA_OPERATOR(split, void, (int* r, split_t* a), ({
    *r = (a->flags) ? a->right : a->left;	
}))

PARACUDA_MAP(split_map, split, int, split_t)

PARACUDA_PERMUTE(int_permute, int)
\end{verbatim}

Den endelige funktion bliver så (variabelnavnene kan
være lidt kryptiske her eftersom vi genbruger vektorene):

\begin{verbatim}
void split(int* array, int* flags, int length)
{
  int* posDown      = PARACUDA_int_allocate_device(length);
  int* posUp        = PARACUDA_int_allocate_device(length);
  int* positions    = PARACUDA_int_allocate_device(length);
  pair_vector_t* pair = PARACUDA_pair_t_shallow_allocate_device();
  split_vector_t* input = PARACUDA_split_t_shallow_allocate_device();

  int before;
  int computed_sum;
  PARACUDA_negate_map_run(posDown, flags, length, PARACUDA_MAX_THREADS);

  PARACUDA_int_peek(&before, posDown, length - 1);
  PARACUDA_plus_scan_run(0, posDown, length, PARACUDA_MAX_THREADS);
  PARACUDA_int_peek(&computed_sum, posDown, length - 1);
  computed_sum += before;

  PARACUDA_int_copy_run(positions, computed_sum, length, PARACUDA_MAX_THREADS);

  PARACUDA_int_copy_device_device(posUp, flags, length);
  PARACUDA_plus_scan_run(0, posUp, length, PARACUDA_MAX_THREADS);

  pair_vector_t host_pair;
  host_pair.x = positions;
  host_pair.y = posUp;
  PARACUDA_pair_t_shallow_copy_host_device(pair, &host_pair);

  PARACUDA_map_add_run(posUp, pair, length, PARACUDA_MAX_THREADS);

  split_vector_t host_input;
  host_input.flags = flags;
  host_input.left = posDown;
  host_input.right = posUp;
  PARACUDA_split_t_shallow_copy_host_device(input, &host_input);

  PARACUDA_split_map_run(positions, input, length, PARACUDA_MAX_THREADS);

  PARACUDA_int_permute_run(posDown, array, positions, length, PARACUDA_MAX_THREADS);

  PARACUDA_int_copy_device_device(array, posDown, length);

  PARACUDA_pair_t_shallow_free_device(pair);
  PARACUDA_split_t_shallow_free_device(input);
  PARACUDA_int_free_device(posDown);
  PARACUDA_int_free_device(posUp);
  PARACUDA_int_free_device(positions);
}
\end{verbatim}

Kildekoden kan iøvrigt findes i bilag \ref{code-split}, 
og datastruktur-, algoritme- og operator-definitionerne
kan findes i \ref{code-kernel}.

\subsection{Radix sort}
\label{radixsort}
Vi har implementeret radix som beskrevet i \cite{ble}, afsnit 3.4.

\[
\begin{array}{l}
\mbox{step} (i, v) = \\
\mathtt{let\ } n = \mbox{copy} (2^i, \#v) \mathtt{\ in}\\
\mathtt{let\ } f = \mbox{map} ((\lambda (x, y).\ x\,\&\,y), \mbox{zip} (n, v)) \mathtt{\ in}\\
\mbox{split} (f, v)
\end{array}
\]

Hvor zip foretages på samme måde som i split, og $\&$ er bitvis eller.
Vi gentager step 32 gange med i fra 0 til 31,
ved hele tiden at anvende step på resultatet af det foregående skridt.

\begin{verbatim}
void radix(int* array, int length, int max_threads)
{
  int* t_numbers = PARACUDA_int_allocate_device(length);
  int* t_flags = PARACUDA_int_allocate_device(length);
  bitwise_vector_t* in = PARACUDA_bitwise_t_shallow_allocate_device();

  for(int i = 0; i < 32; ++i) {
    int num = (1 << i);
    PARACUDA_int_copy_run(t_numbers, num, length, max_threads);

    bitwise_vector_t input;
    input.number = t_numbers;
    input.integer = array;
    PARACUDA_bitwise_t_shallow_copy_host_device(in, &input);

    PARACUDA_int_bitwise_map_run(t_flags, in, length, max_threads);

    split(array, t_flags, length);
  }
  PARACUDA_bitwise_t_shallow_free_device(in);
  PARACUDA_int_free_device(t_numbers);
  PARACUDA_int_free_device(t_flags);
}
\end{verbatim}

Kildekoden kan iøvrigt findes i bilag \ref{code-radix}, 
og datastruktur-, algoritme- og operator-definitionerne
kan findes i \ref{code-kernel}.

\subsection{Forholdet mellem pseudokode og kode}

Sammenligner man pseudo-koden med kildekoden svarer hver linje
i pseudo koden til få linjer C kode - hvis man ser bort fra allokeringen og den separate definitioner
af datastrukturer, operatorer og algoritmer, som alligevel skal gøres lige meget hvilken implementation man bruger.

Det skulle derfor være ganske lige til at omskrive
parallel-vektor-pseudokode til C-kode oven på 
vores bibliotek.
