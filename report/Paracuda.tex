% -*- coding: utf-8 -*-

% NOTER TIL DETTE AFSNIT: (- = to do, + = done, * = note)
% - Gør en pointe ud af at det er generisk for operatorer og data.
%   Men skriv også hvordan det er begrænset (for data).
% - Beskriv hvordan alle primitiverne i Blellochs bog kan bygges
%   ud af de primitiver vi understøtter.
% * Genovervej: hører dette ikke til i bygge-ovenpå-afsnittet?
% - Relater med referencer til teorien.
% + Giv bedre motivation til eksemplet og eventuelt et længere
%   eksempel til sidst.
% + Nævn at makroer bruges og hvordan man skal lave declarations.

\section{Bibliotek}
\label{paracuda}

Biblioteket implementerer en række funktioner bygget på vector-scan-modellen.
Meningen med biblioteket er at man kan bruge det fra C uden at lære CUDA-bibliotekerne at kende.
Dog er der visse begrænsninger på operatorerne som er CUDA-specifikke, og det er ligeledes
nødvendigt at kunne oversætte med CUDA-oversætteren.

Typisk brug af biblioteket består i at definere en datastruktur, en operator og en 
algoritme der bruger disse. Til hver definition bruges en makro fra biblioteket.
Hver algoritme-makro definerer en normal funktion, der kan kaldes som enhver anden 
C-funktion.

Hver algoritme svarer til en primitiv funktion i \cite{ble}, dog med vilkårlige 
brugerdefinerede operatorer og vektorelementtyper.

\subsection{Datastrukturer}

Alle algoritmerne kræver at vi kan kopiere elementer ind og ud af vektorer.
Vi repræsenterer en vektor af datastrukturer som en struct af arrays, da dette
giver os mulighed for at behandle hvert felt i datastrukturen som en individuel
vektor uden at skulle løbe hele vektoren igennem og kopiere det enkelte felt 
over. For at kunne hive et element ud fra en position genererer vi via makroen 
\verb|PARACUDA_STRUCT_N| kode til at hive et enkelt element ud og samle det til 
en struct, og til at splitte et enkelt element op og sætte det ind igen.

Hvis det ønskes kan man også repræsentere vektorer som en array af structs ved
at definere en normal struct og generere kode med \verb|PARACUDA_SINGLE|.

Internt laver vi et alias til vektor-typen med \verb|typedef| 
\verb|PARACUDA_name_struct| hvor \verb|name| er det brugerdefinerede navn for 
datastrukturen. Dermed kan vi slå vektor-typen op hvis vi har navnet for den
ikke-vektoriserede datastruktur, så brugeren kun behøver at specificere denne.

Funktionerne der bliver genereret er navngivet på samme måde og er dækket
i afsnit \ref{manual-memory}. Implementationen kalder CUDA-specifikke 
hukommelsesfunktioner for hvert felt i datastrukturen.

\verb|PARACUDA_STRUCT_2(NAME, VECTOR, T0, X0, T1, X1)| genererer en 
\verb|struct NAME { T0 X0; T1 X1; };| samt en \verb|struct VECTOR { T0* X0; T1* X1; };|.
Den førstenævnte bruges til at håndtere et enkelt element i vektoren, mens den anden bruges til
at håndtere hele vektoren, hvor hvert felt har sin egen array. Der dannes desuden \verb|typedef|s
så de kan bruges i C uden at skulle skrive \verb|struct| foran.

\verb|PARACUDA_SINGLE(TYPE)| genererer funktioner for det angivne typenavn. Hvis typen består
af mere end en token skal der bruges en \verb|typedef|, således at parameteren er en enkelt token.
Det samme navn skal bruges når den endelige algoritme defineres, så makroen kan finde de 
tilknyttede funktioner.

\subsection{Operatorer}

Det vi kalder en operator i vores bibliotek er typisk en funktion der læser fra
et eller flere elementer og skriver til et resultatelement. Det eneste generelle
krav er dog at den er defineret med operator-makroen og overholder CUDAs 
specifikationer for kode der skal køre på grafikkortet (altså ingen rekursion, 
funktions-pointere eller kald til normale funktioner). 
Operator-makroen genererer to versioner af den funktion man specificerer -
en der kan køre på værtssystemet og en der kan køre på grafikkortet.
Sidstenævnte funktion navngives \verb|PARACUDA_name_operator| hvor \verb|name| 
er navnet på operatoren.

\verb|PARACUDA_OPERATOR(NAME, RETURN, ARGUMENTS, BODY)| definerer en normal funktion og en funktion
der kan køre på grafikkortet. \verb|NAME| er funktionens navn, \verb|RETURN| er dens returtype 
(typisk \verb|void|), \verb|ARGUMENTS| er argumenterne (i parentes) og \verb|BODY| er 
funktionskroppen omkranset af paranteser yderst og tuborgklammer inderst
(se eksemplet i afsnit \ref{paracuda-example}).

Operatorer kan hverken bruge rekursion eller funktionspointere, idet disse ikke er understøttet af
CUDA. Desuden kan normale funktioner heller ikke kaldes (kun dem der kan kaldes fra en CUDA 
\verb|__device__| funktion, se eventuelt CUDAs dokumentation).

% NOTE: Det kunne måske være smart hvis man kunne kalde andre operatorer fra operatorer?

\subsection{Algoritmer}

De understøttede algoritmer er herunder angivet med navn, beskrivelse og parametre.

\subsubsection*{\texttt{PARACUDA\_COPY \scriptsize(NAME, TYPE)}}
Definerer en funktion der fylder en vektor af længde $l$ med en værdi.
\[
\mbox{copy}(v, l) = [v, \ldots, v]
\]

\noindent\begin{tabular}{rp{8cm}}
\texttt{NAME} & bliver navnet på den genererede funktion. \\
\texttt{TYPE} & er typen på værdien. \\
\end{tabular}\vspace{0.2cm}

\begin{verbatim}
TYPE-VECTOR* NAME(
        TYPE-VECTOR* out, 
        TYPE in, 
        size_t length, 
        size_t thread_count);
\end{verbatim}

Hvor \verb|TYPE-VECTOR| svarer til vektor-typen for \verb|TYPE|, og \verb|NAME| er det angivne navn. 
Hvis funktionen kaldes med \verb|out=NULL| allokeres hukommelsen 
for \verb|out| automatisk inde i funktionen, og ellers antages det at der allerede er 
allokeret den nødvendige hukommelse. Automatisk allokeret hukommelse vil ikke automatisk
blive deallokeret. Antallet af elementer i inddata angives som \verb|length|. 
Returværdien er en pointer til vektoren med resultatet (\verb|out| med mindre \verb|out=NULL|).
Den sidste parameter er antallet af tråde, og er normalt \verb|PARACUDA_MAX_THREADS|.
Det er tilladt at \verb|out=in| hvis de har samme type.

\subsubsection*{\texttt{PARACUDA\_MAP \scriptsize(NAME, OPERATOR, OUT, IN)}}
Definerer en funktion der implementerer map-algoritmen, som anvender en operator på hvert element.
\[
\mbox{map}(f, [a_0, a_1, \ldots, a_{n-1}]) = [f(a_0), f(a_1), \ldots, f(a_{n-1})]
\]

\noindent\begin{tabular}{rp{8cm}}
\texttt{NAME} & bliver navnet på den genererede funktion. \\
\texttt{OPERATOR} & er den operator som skal anvendes på hvert element.
Den skal tage argumenterne \verb|(OUT* out, IN* in)|. \\
\texttt{OUT} & er typen på uddata. \\
\texttt{IN} & er typen på inddata. \\
\end{tabular}\vspace{0.2cm}

Den genererede funktion kan kaldes som enhver anden funktion og har signaturen

\begin{verbatim}
OUT-VECTOR* NAME(
        OUT-VECTOR* out, 
        IN-VECTOR* in, 
        size_t length, 
        size_t thread_count);
\end{verbatim}

Hvor \verb|OUT-VECTOR| og \verb|IN-VECTOR| svarer til vektor-typerne for henholdsvis
\verb|OUT| og \verb|IN|, og \verb|NAME| er det angivne navn. 
Hvis funktionen kaldes med \verb|out=NULL| allokeres hukommelsen 
for \verb|out| automatisk inde i funktionen, og ellers antages det at der allerede er 
allokeret den nødvendige hukommelse. Automatisk allokeret hukommelse vil ikke automatisk
blive deallokeret. Antallet af elementer i inddata angives som \verb|length|. 
Returværdien er en pointer til vektoren med resultatet (\verb|out| med mindre \verb|out=NULL|).
Den sidste parameter er antallet af tråde, og er normalt \verb|PARACUDA_MAX_THREADS|.
Det er tilladt at \verb|out=in| hvis de har samme type.

\subsubsection*{\texttt{PARACUDA\_SCAN \scriptsize(NAME, OPERATOR, NEUTRAL, TYPE)}}
Definerer en funktion der implementerer scan-algoritmen (exclusive prefix sum),
som er i familie med fold.
\[
\mbox{scan}(\oplus, z, [a_0, a_1, \ldots, a_{n-1}]) = [z, a_0, a_0 \oplus a_1, \ldots, a_0 \oplus a_1 \oplus \cdots \oplus a_{n-2}]
\]

\noindent\begin{tabular}{rp{8cm}}
\texttt{NAME} & bliver navnet på den genererede funktion. \\
\texttt{OPERATOR} & er den operator der skal anvendes på element-par.
Det er et krav at operatoren er associativ, så det er ligegyldigt hvor man sætter paranteserne: 
$(a\oplus b)\oplus c = a\oplus (b\oplus c)$, og at den sammen med det neutrale element
(og typen) danner en monoid. Den skal tage argumenterne 
\verb|(TYPE* result, TYPE* left, TYPE* right)|. \\
\texttt{NEUTRAL} & er en operator der returnerer det neutrale element. 
Den skal tage argumenterne 
\verb|(TYPE* result)|. \\
\texttt{TYPE} & er typen på ud- og inddata. \\
\end{tabular}\vspace{0.2cm}

Den genererede funktion kan kaldes som enhver anden funktion og har signaturen

\begin{verbatim}
TYPE-VECTOR* NAME(
        TYPE* sum, 
        TYPE-VECTOR* out, 
        TYPE-VECTOR* in, 
        size_t length, 
        size_t thread_count);
\end{verbatim}

Hvor \verb|TYPE-VECTOR| svarer til vektor-typen for \verb|TYPE|, og \verb|NAME| er det angivne navn. 
Hvis funktionen kaldes med \verb|out=NULL| allokeres hukommelsen 
for \verb|out| automatisk inde i funktionen, og ellers antages det at der allerede er 
allokeret den nødvendige hukommelse. Automatisk allokeret hukommelse vil ikke automatisk
blive deallokeret. Antallet af elementer i inddata angives som \verb|length|. 
Hvis \verb|sum| er forskellig fra \verb|NULL| lagres summen af alle elementerne i denne:
$a_0 \oplus a_1 \oplus \cdots \oplus a_{n-1}$.
Returværdien er en pointer til vektoren med resultatet (\verb|out| med mindre \verb|out=NULL|).
Den sidste parameter er antallet af tråde, og er normalt \verb|PARACUDA_MAX_THREADS|.
Det er tilladt at \verb|out=in|.

\subsubsection*{\texttt{PARACUDA\_SEGMENTEDSCAN \scriptsize(NAME, OPERATOR, NEUTRAL, TYPE)}}
Definerer en funktion der implementerer segmented scan. Hvert sat flag indikerer starten
på en ny delvektor.
\[
\begin{array}{l}
\mbox{scan}(\oplus, z, [a_0, a_1, a_3, a_4, a_5, a_6], [0, 0, 0, 1, 0, 0]) =
[z, a_0, a_0 \oplus a_1, z, a_3, a_3 \oplus a_4]
\end{array}
\]

Den genererede funktion kan kaldes som enhver anden funktion og har signaturen

\begin{verbatim}
TYPE-VECTOR* NAME(
        TYPE-VECTOR* out, 
        TYPE-VECTOR* in, 
        int* flags, 
        size_t length, 
        size_t thread_count);
\end{verbatim}

Selvom der ikke er nogen entydig sum  i Segmented Scan, så skal resultaterne fra de enkelte logiske arrays stadigvæk gemmes i en vektor til senere brug - f.eks. hvis Segmented Split skulle implementeres. Udover dette, flagene og 
opdelingen i delvektorer er den identisk med \verb|PARACUDA_SCAN|.

\subsubsection*{\texttt{PARACUDA\_PERMUTE \scriptsize(NAME, TYPE)}}
Definerer en funktion der returnerer en given permutation af en vektor.
\[
\mbox{permute}([a_0, a_1, \ldots, a_{n-1}], [i_0, i_1, \ldots, i_{n-1}]) = [a_{i_0}, a_{i_1}, \ldots, a_{i_{n-1}}]
\]
\noindent\begin{tabular}{rp{8cm}}
\texttt{NAME} & bliver navnet på den genererede funktion. \\
\texttt{TYPE} & er typen på ud- og inddata. \\
\end{tabular}\vspace{0.2cm}

Den genererede funktion kan kaldes som enhver anden funktion og har signaturen

\begin{verbatim}
TYPE-VECTOR* NAME(
        TYPE-VECTOR* out, 
        TYPE-VECTOR* in, 
        int* positions, 
        size_t length, 
        size_t thread_count);
\end{verbatim}

Hvor \verb|TYPE-VECTOR| svarer til vektor-typen for \verb|TYPE|, og \verb|NAME| er det angivne navn. 
Hvis funktionen kaldes med \verb|out=NULL| allokeres hukommelsen 
for \verb|out| automatisk inde i funktionen, og ellers antages det at der allerede er 
allokeret den nødvendige hukommelse. Automatisk allokeret hukommelse vil ikke automatisk
blive deallokeret. Antallet af elementer i inddata angives som \verb|length|. 
Permutationen angives som en vektor af positioner \verb|positions|, og det antages at den samme
position ikke optræder flere gange.
Returværdien er en pointer til vektoren med resultatet (\verb|out| med mindre \verb|out=NULL|).
Den sidste parameter er antallet af tråde, og er normalt \verb|PARACUDA_MAX_THREADS|.
Det er tilladt at \verb|out=in|.

\subsection{Eksempel}
\label{paracuda-example}

Lad os sige vi har brug for at definere følgende funktion:
\[
\mbox{plus\_times\_map } [(a_0, b_0), (a_1, b_1), \ldots] = 
[(a_0 + b_0, a_0 \times b_0), (a_1 + b_1, a_1 \times b_1), \ldots]
\]
Dette kræver en par-datastruktur, en $+\times$-operator og map-algoritmen.

\begin{verbatim}
#include "paracuda.h"

PARACUDA_STRUCT_2(pair_t, pair_vector_t,
    int, x,
    int, y
)

PARACUDA_OPERATOR(plus_times, void, (pair_t* r, pair_t* v), ({
    r->x = v->x + v->y; 
    r->y = v->x * v->y;
}))

PARACUDA_MAP(plus_times_map, plus_times, pair_t, pair_t)
\end{verbatim}

Ovenstående skal gemmes i en \verb|.cu| (CUDA) fil og skal oversættes med 
cuda-oversætteren.

Den genererede funktion \verb|plus_times_map| kan 
kaldes som en normal funktion og har følgende signatur:

\begin{verbatim}
pair_vector_t* plus_times_map(
        pair_vector_t* out, 
        pair_vector_t* in, 
        size_t length, 
        size_t thread_count);
\end{verbatim}

\subsection{Manuel håndtering af hukommelse}
\label{manual-memory}

Alle funktioner der genereres med algoritme-makroer kopierer først vektorerne ned i grafikkortets
hukommelse, og derefter tilbage igen når udregningen er færdig. Man kan undgå dette ved at bruge
følgende funktioner, hvor \verb|T| er navnet på datastrukturen og \verb|F| er navnet på funktionen.

Det er ikke tilladt at tilgå hukommelse der ligger på grafikkortet uden disse funktioner.
Overholdes dette ikke er programmets opførsel ikke veldefineret, men en typisk fejlmeddelelse
er \verb|segmentation fault|.

\subsubsection*{\texttt{PARACUDA\_T\_allocate\_host \scriptsize(length)}}

Denne funktion allokerer og returnerer en vektor af typen \verb|T| med længden \verb|length| i værtssystemet.

\subsubsection*{\texttt{PARACUDA\_T\_allocate\_device \scriptsize(length)}}

Denne funktion allokerer og returnerer en vektor af typen \verb|T| med længden \verb|length| på grafikkortet.

\subsubsection*{\texttt{PARACUDA\_T\_copy\_host\_device \scriptsize(out, in, length)}}

Denne funktion kopierer en vektor fra værtssystemet til grafikkortet.

\subsubsection*{\texttt{PARACUDA\_T\_copy\_device\_host \scriptsize(out, in, length)}}

Denne funktion kopierer en vektor fra grafikkortet til værtssystemet.

\subsubsection*{\texttt{PARACUDA\_T\_copy\_device\_device \scriptsize(out, in, length)}}

Denne funktion kopierer en vektor fra et sted på grafikkortet til et andet.

\subsubsection*{\texttt{PARACUDA\_T\_from\_vector \scriptsize(out, in, index)}}

Kopierer et element ud af en vektor og ind i \verb|out|, der er af typen \verb|T|.

\subsubsection*{\texttt{PARACUDA\_T\_to\_vector \scriptsize(out, in, index)}}

Kopierer et element ind i en vektor fra \verb|in|, der er af typen \verb|T|.

\subsubsection*{\texttt{PARACUDA\_T\_shallow\_allocate\_host \scriptsize(out, in, index)}}

Allokerer en vektor på værtssystemet uden at allokere hukommelse til indholdet.

\subsubsection*{\texttt{PARACUDA\_T\_shallow\_allocate\_device \scriptsize(out, in, index)}}

Allokerer en vektor på grafikkortet uden at allokere hukommelse til indholdet.

\subsubsection*{\texttt{PARACUDA\_T\_shallow\_copy\_host\_device \scriptsize(out, in, index)}}

Kopierer en vektor til grafikkortet under den antagelse at dens pointere allerede peger
på data på grafikkortet. Indholdet bliver ikke kopieret.

\subsubsection*{\texttt{PARACUDA\_T\_shallow\_copy\_device\_host \scriptsize(out, in, index)}}

Kopierer en vektor til værtsystemet uden at kopiere indholdet, så dens pointere stadig
peger på data på grafikkortet.

\subsubsection*{\texttt{PARACUDA\_F\_run \scriptsize(...)}}

Denne funktion kalder funktionen \verb|F| under den antagelse at alle vektorer allerede
ligger på grafikkortet. Resultatet skal manuelt kopieres tilbage til værtssystemet.
Hukommelse til uddata allokeres ikke automatisk. Der er små variationer i 
funktionssignaturen i forhold til \verb|F|. Disse ses bedst i kildekoden for den
individuelle \verb|run|-funktion og dens tilknyttede kommentar.
